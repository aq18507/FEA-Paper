\documentclass[a4paper]{article}

% Standard Package
\usepackage[margin=25mm]{geometry}
\usepackage[numbers]{natbib}
\usepackage{amsmath}
\usepackage{amsfonts}
\usepackage{amssymb}
\usepackage{graphicx}
\pagenumbering{gobble}
\usepackage{verbatim}
\immediate\write18{texcount -tex -sum  \jobname.tex > \jobname.wordcount.tex}

% Keywords command
\providecommand{\keywords}[1]
{
  \small	
  \textbf{\textit{Keywords---}} #1
}

% \title{Out of Wind Tunnel Differential Pressure Performance of 3D Printed GATOR Morphing Aircraft Skins: An Experimental and FEA Study}
\title{Differential Pressure Analysis Of 3D Printed GATOR Morphing Skins: An FEA And Novel Experimental Approach}
\author{Rafael Martin Heeb$^{1}$, Benjamin King Sutton Woods$^{1}$  \\
        \small $^{1}$University of Bristol\\
        \small Department of Aerospace Engineering\\
        \small Queens Building\\
        \small Bristol\\
        \small BS8 1TR\\
        \small United Kingdom \\
}
\date{\today} % Comment this line to show today's date

\begin{document}
\maketitle

\begin{abstract}
    
    % Morphing aircraft are a promising solution to reducing the aerospace industry's environmental impact, reducing the induced drag and thus the fuel burn. One of the main reasons why morphing aircraft have not yet been widely adopted is the competing design requirement of morphing skins. They have to have a high out-of-plane stiffness to resist aerodynamic loading, whilst having a low in-plane stiffness to keep the actuation forces low, all of this whilst offering a smooth aerodynamic surface and being a lightweight structure.\medskip

    % The novel concept of 3D printed Geometrically Anisotropic ThermOplasic Rubber (GATOR) morphing aircraft skins seeks to solve this problem. It utilizes three main principles, which sets it apart from other methods: 1) Multi-material 3D printers which are capable of different materials in the same manufacturing process, 2) printing different formulations of Thermoplastic Polyurethane (TPU) which have different stiffnesses, 3) and the exploitation of geometric scaling laws and anisotropy to better balance in-plane and out-of-plane mechanical properties. Experimental research by these authors has shown a significant reduction in the complexity of the morphing skin, where stiff stringers were printed directly onto the skin sheet with a perfect bond between the two formulations of TPU. This was attached to a 1.0m long FishBAC morphing trailing edge, and wind tunnel tests have indicated that there is no reduction in aerodynamic performance\cite{Rivero2022}.\medskip

    % To exploit the geometric scaling laws, the GATOR methods were used where a zero Poisson's ratio MoprhCore was printed directly onto a highly flexible skin sheet, using a stiff and a soft formulation of TPU, respectively. The core offers the out-of-plane stiffness, where the skin provides, a smooth aerodynamic surface creating a lightweight morphing skin made in one single manufacturing process, strategically printing materials to design against the competing structural requirements. As part of the structural exploration of this particular configuration, a novel static method was developed to evaluate the differential pressure performance of 3D printed GATOR morphing skins at a range of stretch ratios from 1 to 1.6, negating the use of complex and expensive wind tunnel tests. This method utilizes the full advantage of additive manufacturing, where one single panel was printed, which was folded around two end plates and the open surface welded together creating an airtight inflatable container with four identical surfaces. The end plates were then attached to a Shimadzu EZ tensile tester in order to stretch the specimen, where the end plates also contained a pressure sensor and pressure supply port. The surfaces were speckled, using acrylic paint to capture the complex in- and out-of-plane deformation mechanism of the skin sheet using Digital Imaging Correlation (DIC) and the internal pressure was monitored using the pressure sensor. In addition to this, a full scale FEA model was developed using hyperelastic material properties derived in a previous study by these authors. The specimen was stretched up to a stretch ratio of 1.6 and tested at a mean differential pressure of 0.022MPa.\medskip

    % % Results
    % This experiment shows some promising results. First and formeost, it is possible to not only stretch the specimen up to a stretch ratio of 1.6, but it is also able to keep the pressure over the entire extension cycle. More importantly the DIC data showed that there is some local out-plane deformation of the unsupported skin sheet up to a stretch ratio of 1.1. As a result the overall out-of-plane deformation reduces by 11.2\% but at stretch ratios higher than 1.1, the out-of-plane deformation starts to increase again up to 12.3\% due to the increased surface of the pressurized area. The FEA model shows a very promising tool to predict the deformation pattern of such a GATOR skin.

    Morphing aircraft present a potential solution for minimizing the aerospace industry's environmental footprint by decreasing induced drag and, as a result fuel consumption. A primary barrier to their widespread adoption is the competing design requirements of morphing skins. These skins must possess high out-of-plane stiffness to counteract aerodynamic loads, but simultaneously maintain low in-plane stiffness to minimize actuation forces. Moreover, they must provide a smooth aerodynamic surface while remaining a lightweight structure.

    Addressing this challenge, a novel concept was introduced by these authors: 3D printed Geometrically Anisotropic ThermOplasic Rubber (GATOR) morphing aircraft skins. It utilizes three main principles, which sets it apart from other methods: 1) the use of multi-material 3D printers, which allow multiple materials to be printed simultaneously, 2) employing multiple formulations of Thermoplastic Polyurethane (TPU) with varying stiffness properties, and 3) exploitation geometric scaling laws and anisotropy to optimize in-plane and out-of-plane mechanical properties. This research evidences a substantial simplification in morphing skin design. Stiff stringers were directly printed onto the skin sheet, creating an impeccable bond between different TPU formulations negating the complex assembly of the aerodynamic skin sheet to the underlying structure. When applied to a 1.0m long FishBAC morphing trailing edge, wind tunnel tests confirmed no decrement in aerodynamic efficiency.
    
    Capitalizing on geometric scaling principles, the GATOR technique printed a zero Poisson's ratio MorphCore directly onto a highly flexible skin sheet using both stiff and soft TPU formulations respectively. The core imparts out-of-plane stiffness, while the skin ensures a smooth aerodynamic surface, collectively producing a lightweight morphing skin in a singular manufacturing process. This tailored printing addresses the conflicting structural prerequisites. As part of this research a novel static method to assess the differential pressure performance of 3D printed GATOR skins across stretch ratios of 1 to 1.6 was developed, eliminating the need for complicated, time-consuming and expensive wind tunnel testing. This method fully harnesses the power of additive manufacturing. A single panel was printed, subsequently folded around two end plates and sealed, forming an airtight inflatable chamber. The end plates, attached to a Shimadzu EZ tensile tester for specimen stretching, incorporated both a pressure sensor and supply port. Digital Imaging Correlation (DIC) was employed to record intricate skin sheet deformations, while internal pressure variations were monitored via the pressure sensor. In tandem, a comprehensive high-fidelity FEA model was developed, based on hyperelastic material properties outlined in our previous studies using 3D solid and 2D shell elements to model the core and skin sheet respectively.
    
    These experiments yielded encouraging outcomes. The specimen could be stretched to a ratio of 1.6 while maintaining consistent pressure throughout the extension cycle with only a limited amount of leakage. Notably, the DIC data revealed local out-of-plane deformation of the unsupported skin sheet up to a ratio of 1.1. Consequently, overall out-of-plane deformation decreased by 11.2\%. However, for stretch ratios exceeding 1.1, the out-of-plane deformation increased again to 12.3\% due to the enlarged pressurized surface area. The DIC data furthermore showed that core deformation pattern projected onto the skin sheet results in a local strain amplification effect, which as a result can reduce the amount of pre-tension required to reduce the local ballooning effect. The FEA model demonstrated its efficacy in predicting the deformation patterns of the GATOR skin with a very complex core behaviour caused by pre-strained and unstained material properties.
    

\end{abstract} \hspace{10pt}

% Keywords
\keywords{Morphing, Adaptive Structures, Digital Imaging Correlation, Finite Element Analysis, Static Testing, Morphing Aircraft Skins, Additive manufacturing}


% Bibliography
\bibliographystyle{agsm}
\bibliography{Research_Library}

% Word count
\verbatiminput{\jobname.wordcount.tex}

\end{document}